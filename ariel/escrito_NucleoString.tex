\documentclass[10pt,a4paper,spanish] {article}
\usepackage[spanish]{babel}
\usepackage[T1]{fontenc}
\usepackage[utf8]{inputenc}
\usepackage{graphicx}
\usepackage{tikz}
\usepackage{etex}


\begin{document}
\title{Nucleotide String Library}
\author{Juan José Delgado Quesada B42250\\José Ariel  Fallas Pizarro B42481\\David Martínez García B34019}

\maketitle

\section{Justificación}

Los nucleótidos son la base de nuestros genes. Estos se unen de manera única y específica para crear cada uno de los genes. Estos son únicos y se componen unicamente de 4 tipos de nucleótidos (Adenina, Guanina, Citosina, Timina). 
Una pequeña modificación en el código genético puede significarse una mutación, la cual podría terminar generando una especie completamente distinta a la originaria. 
Y el ADN de un organismo se compone de millones de millones de genes y comparar dos de ellos podría tomarse semanas e inclusive meses si se hace manualmente. Es por esto que automatizar este proceso es necesario, ya que si queremos descubrir una nueva especie o averiguar con cual especie de bacteria estamos tratando por ejemplo, podemos tomar una muestra de esta y compararla con una base de datos ya existente y hacerlo en cuestión de minutos

\section{Objetivos General}

\begin{itemize}
\item Implementar una librer\'ia en C++ que facilite el trabajo y procesamiento de secuencias de bases nitrogenadas y a su vez permita trabajar con bases de datos de cadenas de bases nitrogenadas \textbf{(BLAST)}.
\end{itemize}

\subsection{Objetivos Específicos}
\begin{itemize}
\item Diseñar y estructurar las clases necesarias, que conformaran la Libreria para el manejo de bases nitrogenadas, con sus respectivas funciones.
\item Desarrollar una aplicación que permita poner en práctica las funciones y clases perteneciente a la libreria para el manejo de bases nitrogenadas.
\item Poner en práctica los conocimientos vistos en el curso de Estructuras Abstractas de Datos y Algoritmos para Ingeniería como lo son: el uso de Templates y el análisis de la efectividad.

\end{itemize}

\end{document}
\documentclass[10pt,a4paper,spanish] {article}
\usepackage[spanish]{babel}
\usepackage[T1]{fontenc}
\usepackage[utf8]{inputenc}
\usepackage{graphicx}
\usepackage{tikz}
\usepackage{etex}


\begin{document}
\title{documento ejemplo}
\authors{José Ariel Fallas, David Martinez, Juan José Gonzales}

\maketitle

\section{Justificación}

Los nucleótidos son la base de nuestros genes. Estos se unen de manera única y específica para crear cada uno de los genes. Estos son únicos y se componen unicamente de 4 tipos de nucleótidos (Adenina, Guanina, Citosina, Timina). 
Una pequeña modificación en el código genético puede significarse una mutación, la cual podría terminar generando una especie completamente distinta a la originaria. 
Y el ADN de un organismo se compone de millones de millones de genes y comparar dos de ellos podría tomarse semanas e inclusive meses si se hace manualmente. Es por esto que automatizar este proceso es necesario, ya que si queremos descubrir una nueva especie o averiguar con cual especie de bacteria estamos tratando por ejemplo, podemos tomar una muestra de esta y compararla con una base de datos ya existente y hacerlo en cuestión de minutos

\section{Objetivos General}

\begin{itemize}
\item Crear una librería capaz de trabajar en el procesamiento de cadenas de nucleótidos para facilitar y agilizar el trabajo de los investigadores del centro de investigación en biología celular y molecular \textbf{(CIBCM)}
\end{itemize}

\section{Objetivos Específicos}
\begin{itemize}
\item Crear una clase capaz  de procesar información sobre nucleótidos  
\item Crear una clase que se comunique con una base de datos en internet

\section{•}



\end{itemize}

